\chapter{まとめと今後の展望}
本論文では,幾何学的な位相に注目し,Berry位相やAharonov-Anandan位相,非断熱遷移における幾何学的位相,測地曲率が,物理的な系において重要な役割を担うことを説明した.特に,Twisted Landau-Zenerモデルにおいて,測地曲率が非断熱遷移の確率に影響を与えることは,本論文のメインテーマである.また,Landau-Zenerモデルを含むcyclic evolutionという具体例を用いて,波動関数の干渉によって状態の占有確率が変化することを確認した.さらに,Twisted Landau-Zener遷移を含むcyclic evolutionを新たに考案した.そのうえで,この系において,波動関数の干渉と測地曲率が織りなす現象をみることができた.特に,完全トンネルと整流作用という測地曲率の幾何学的効果によって,2回の遷移ごとに状態の占有確率が変化する現象は,従来のcyclic evolutionでは見られない.


今後は,Twisted Landau-Zener遷移を含むcyclic evolutionにおいて,波動関数の干渉と測地曲率の関係を解析的に調べるため,波動関数の解析解を導出したい.具体的には,MajoranaやZenerがLandau-Zenerモデルに対して行った,Laplace変換や微分方程式による方法を用いて,Twisted Landau-Zenerモデルにおける波動関数の解析解の導出を模索する.仮にそれが可能であれば,転送行列法における転送行列が決定できるはずであるため,Twisted Landau-Zener遷移を含むcyclic evolutionの波動関数も導出する.また,本論文では,cyclic evolutionにおけるAharonov-Anandan位相と非断熱遷移の幾何学的位相の関係にまったく踏み込むことができなかった.Aharonov-Anandan位相は,接続の微分幾何やゲージ理論を用いて体系的に理解することができる.このように,数学的な観点から,本論文で登場したさまざまな幾何学的な量の関係を明らかにすることも今後の課題としておく.


\section*{謝辞}
\acknowledgement